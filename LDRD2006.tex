\documentclass[pdf,12pt,report,strict]{SANDreport}

\usepackage{amsfonts}
\usepackage{amssymb}
\usepackage{graphicx}
\usepackage{varioref}
\usepackage{fancyvrb}
\usepackage{cite}
\usepackage{subfigure}
\usepackage{xspace}
\usepackage{hyperref}

%-----------------------------------------------------------------------------
% My commands
%
\newcommand*{\lcite}[1]{~\cite{#1}}
\newcommand*{\scite}[1]{~\cite{#1}}
\newcommand{\titan}{Titan}
\newcommand{\threatview}{ThreatView\texttrademark\xspace}

% ---------------------------------------------------------------------------- %
% Set the title, author, and date
%
\title{Massive Graph Visualization}
\author{Dr. Kenneth D. Moreland, SMTS\\
Scalable Analytics \& Visualization\\
P.O. Box 5800\\
Albuquerque, NM 87185-1323\\
\\
Brian N. Wylie, PMTS\\
Scalable Analytics \& Visualization\\
P.O. Box 5800\\
Albuquerque, NM 87185-1323}
\date{}					% Leave this here but empty


% ---------------------------------------------------------------------------- %
% These are mandatory
%
\SANDnum{SAND 2007-XXXX}			% e.g. \SANDnum{SAND2006-0420}
\SANDprintDate{}
\SANDauthor{Kenneth~Moreland and Brian~Wylie}	% One line, separated by commas


% ---------------------------------------------------------------------------- %
% These are optional
%
%\SANDrePrintDate{}	% May be repeated for successive printings
%\SANDsupersed{}{}	% {Old SAND number}{Old date}


% ---------------------------------------------------------------------------- %
% Build your markings. See example files and SAND Report Guide
%
% \SANDreleaseType{}
% \SANDmarkTopBottomCoverBackTitle{}
% \SANDmarkBottomCover{}
% \SANDmarkTopBottomCoverTitle{}
% \SANDmarkTop{}
% \SANDmarkBottom{}
% \SANDmarkTopBottom{}
% \SANDmarkCover{}
% \SANDmarkCoverTitle{}


% ---------------------------------------------------------------------------- %
% Start the document
%
\begin{document}
\maketitle

% ------------------------------------------------------------------------ %
% An Abstract is required for SAND reports
% 
\begin{abstract}
  Graphs are a vital way of organizing data with complex correlations. A
  good visualization of a graph can fundamentally change human
  understanding of the data. Consequently, there is a rich body of work on
  graph visualization.  Although there are many techniques that are
  effective on small to medium sized graphs (tens of thousands of nodes),
  there is a void in the research for visualizing massive graphs containing
  millions of nodes. Sandia is one of the few entities in the world that
  has the means and motivation to handle data on such a massive scale. For
  example, homeland security generates graphs from prolific media sources
  such as television, telephone, and the Internet.  The purpose of this
  project is to provide the groundwork for visualizing such massive graphs.
  The research provides for two major feature gaps: a parallel, interactive
  visualization framework and scalable algorithms to make the framework
  usable to a practical application.  Both the frameworks and algorithms
  are designed to run on distributed parallel computers, which are already
  available at Sandia.  Future work will integrate these features into the
  \threatview application.
\end{abstract}


% ------------------------------------------------------------------------ %
% An Acknowledgment section is optional but important
% 
\clearpage
\chapter*{Acknowledgment}

The Massive Graph Visualization LDRD team would like to acknowledge the
significant support provided by Nabeel Rahal, our LDRD program manager.
Without Nabeel's fanatical support, our work may never have left the
ground.  We would like to acknowledge those who provided both direct
and indirect technical development: Timothy Shead and Jeffrey Baumes.  We
also acknowledge Bruce Hendrickson, Jonathan Berry, and Patricia Crossno
for their technical guidance.


% ------------------------------------------------------------------------ %
% The table of contents and list of figures and tables
% 
\cleardoublepage		% TOC needs to start on an odd page
\tableofcontents
\listoffigures
%\listoftables


%% % ---------------------------------------------------------------------- %
%% % An optional preface or Foreword
%% \clearpage
%% \chapter*{Preface}
%% \addcontentsline{toc}{chapter}{Preface}


% ---------------------------------------------------------------------- %
% An optional executive summary
\cleardoublepage		% Executive Summary to start on odd page
\chapter*{Executive Summary}
\addcontentsline{toc}{chapter}{Executive Summary}


%% % ---------------------------------------------------------------------- %
%% % An optional glossary. We don't want it to be numbered
%% \clearpage
%% \chapter*{Nomenclature}
%% \addcontentsline{toc}{chapter}{Nomenclature}
%% \begin{description}
%% \item[Term 1]
%%   Description
%% \item[Term 2]
%%   Description
%% \item[Term 3]
%%   Description
%% \end{description}


% ---------------------------------------------------------------------- %
% This is where the body of the report begins; usually with an Introduction
% 
\SANDmain		% Start the main part of the report

\chapter{Introduction}
\label{chap:Introduction}

\section{Related Projects}
\label{sec:RelatedProjects}

\subsection{\titan}
\label{sec:RelatedProjects:Titan}

\subsection{\threatview}
\label{sec:RelatedProjects:ThreatView}

\subsection{Multi-Threaded Graph Library}
\label{sec:RelatedProjects:MTGL}

\subsection{VxOrd}
\label{sec:RelatedProjects:VxOrd}

Structure of science stuff\lcite{Boyak04,Boyak05}.


\chapter{Parallel Graph Visualization Framework}
\label{sec:ParallelGraphVisualizationFramework}

\section{Data Structures}
\label{sec:ParallelGraphVisualizationFramework:DataStructures}

2D edge partitioning\lcite{Yoo05} is cool.

\section{Reading and Querying}
\label{sec:ParallelGraphVisualizationFramework:ReadingAndQuerying}


\chapter{Parallel Graph Visualization Algorithms}
\label{chap:ParallelGraphVisualizationAlgorithms}

\section{Layout}
\label{sec:ParallelGraphVisualizationAlgorithms:Layout}

\subsection{Fast Layout}
\label{sec:ParallelGraphVisualizationAlgorithms:Layout:FastLayout}

\subsection{G-Space}
\label{sec:ParallelGraphVisualizationAlgorithms:Layout:GSpace}

\section{Edge Lighting}
\label{sec:ParallelGraphVisualizationAlgorithms:EdgeLighting}

\section{Landscape View}
\label{sec:ParallelGraphVisualizationAlgorithms:LandscapeView}


\chapter{Future Work}
\label{chap:FutureWork}


%% \chapter{Junk}

%% \begin{table}[ht]
%%   \centering
%%   \caption[Short Title]{Full caption}
%%   \bigskip

%%   \begin{tabular}{|l|c|l|c|}
%%   \end{tabular}
%%   \label{tab:1}
%% \end{table}

%% \begin{figure}[ht]
%%   \centering
%%   \subfigure[Short title]{
%%     \label{fig:sub:1}
%% %    \includegraphics[keepaspectratio=true, width= in]{filename}
%%   }
%%   \subfigure[Short title]{
%%     \label{fig:sub:2}
%% %    \includegraphics[keepaspectratio=true, width= in]{filename}
%%   }
%%   \caption{Full caption.}
%%   \label{fig:1}
%% \end{figure}



% ---------------------------------------------------------------------- %
% References
% 
\clearpage
% If hyperref is included, then \phantomsection is already defined.
% If not, we need to define it.
\providecommand*{\phantomsection}{}
\phantomsection
\addcontentsline{toc}{chapter}{References}
\bibliographystyle{plain}
\bibliography{LDRD2006}


%% % ---------------------------------------------------------------------- %
%% % 
%% \appendix
%% \chapter{}

% \printindex

\begin{SANDdistribution}[NM]% or [CA]
  % \SANDdistCRADA	% If this report is about CRADA work
  % \SANDdistPatent	% If this report has a Patent Caution or Patent Interest
  % \SANDdistLDRD	% If this report is about LDRD work

  % External Address Format: {num copies}{Address}
  \SANDdistExternal{}{}
  \bigskip

  % The following MUST BE between the external and internal distributions!
  % \SANDdistClassified % If this report is classified

  % Internal Address Format: {num copies}{Mail stop}{Name}{Org}
  \SANDdistInternal{}{}{}{}

  % Mail Channel Address Format: {num copies}{Mail Channel}{Name}{Org}
  \SANDdistInternalM{}{}{}{}
\end{SANDdistribution}


% The second printing
% \begin{SANDreDistribution}
%   \SANDdistExternal{}{}
%   \bigskip
%   \SANDdistInternal{}{}{}{}
%   \SANDdistInternalM{}{}{}{}
% \end{SANDreDistribution}

\end{document}
